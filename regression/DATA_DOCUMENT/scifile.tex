% Use only LaTeX2e, calling the article.cls class and 12-point type.

\documentclass[12pt]{article}

% Users of the {thebibliography} environment or BibTeX should use the
% scicite.sty package, downloadable from *Science* at
% www.sciencemag.org/about/authors/prep/TeX_help/ .
% This package should properly format in-text
% reference calls and reference-list numbers.

\usepackage{scicite}

% Use times if you have the font installed; otherwise, comment out the
% following line.

\usepackage{times}
\usepackage{hyperref}

% The preamble here sets up a lot of new/revised commands and
% environments.  It's annoying, but please do *not* try to strip these
% out into a separate .sty file (which could lead to the loss of some
% information when we convert the file to other formats).  Instead, keep
% them in the preamble of your main LaTeX source file.


% The following parameters seem to provide a reasonable page setup.

\topmargin 0.0cm
\oddsidemargin 0.2cm
\textwidth 16cm 
\textheight 21cm
\footskip 1.0cm


%The next command sets up an environment for the abstract to your paper.

\newenvironment{sciabstract}{%
\begin{quote} \bf}
{\end{quote}}


% If your reference list includes text notes as well as references,
% include the following line; otherwise, comment it out.

\renewcommand\refname{References and Notes}

% The following lines set up an environment for the last note in the
% reference list, which commonly includes acknowledgments of funding,
% help, etc.  It's intended for users of BibTeX or the {thebibliography}
% environment.  Users who are hand-coding their references at the end
% using a list environment such as {enumerate} can simply add another
% item at the end, and it will be numbered automatically.

\newcounter{lastnote}
\newenvironment{scilastnote}{%
\setcounter{lastnote}{\value{enumiv}}%
\addtocounter{lastnote}{+1}%
\begin{list}%
{\arabic{lastnote}.}
{\setlength{\leftmargin}{.22in}}
{\setlength{\labelsep}{.5em}}}
{\end{list}}


% Include your paper's title here

\title{Jobsearch: or where to go to find an internship!} 


% Place the author information here.  Please hand-code the contact
% information and notecalls; do *not* use \footnote commands.  Let the
% author contact information appear immediately below the author names
% as shown.  We would also prefer that you don't change the type-size
% settings shown here.

\author
{Antonio Curado,$^{1}$ Joris Bertens,$^{2}$ Manuel Demetriades,$^{3}$ Morten Dahl$^{4}$\\
\\
\normalsize{$^{1}$M20180032@novaims.unl.pt}\\
\normalsize{$^{2}$M20180423@novaims.unl.pt}\\
\normalsize{$^{3}$M20180425@novaims.unl.pt}\\
\normalsize{$^{4}$M20180047@novaims.unl.pt}\\
}

% Include the date command, but leave its argument blank.

\date{}



%%%%%%%%%%%%%%%%% END OF PREAMBLE %%%%%%%%%%%%%%%%



\begin{document} 

% Double-space the manuscript.

\baselineskip20pt

% Make the title.

\maketitle 



% Place your abstract within the special {sciabstract} environment.

\begin{sciabstract}
  The key objective of our project is to investigate and identify key factors affecting portuguese unemployment rates at a municipal level. In addition to implementing a predictive system for unemployment rates, we aim to attain further insight to the underlying factors identified, providing key information for municipalities and their projected path. 

\end{sciabstract}



\section*{Issue}



{Occupational opportunities are limited and an issue in Portugal. Hence, it should be at local governments best interests to bring more local businesses to their municipalities to create more job opportunities and economic growth.
}.


\section*{Solution}

{Identify and scrutinise factors exhibiting significant influence on (un)employment rates, utilising them for the prediction of “investment strategies” by local governance.
}

\section*{Data}

{The data retrieved and utilized to compose this dataset were taken from pordata, the database of Contempory Portugal. All data was retrieved on a municipality level and joined on the unique municipality. The collected is exclusively from year 2011. As in this year the census took place the data is particurlaly versatile and consists of many features. The data can be retrieved \href{https://drive.google.com/open?id=1oQgpETAuAHQqdlWaiQMgGuqTx9VxdCZY}{here}.

The collected data consists of relevant factors and are described below.

\subsection*{\textit{Composition of Unemployment}}
In this section data was taken describing the composition of unemployment based on age groups, education per sex and job seek per sex (a binary variable saying if the person seeks a job for the first time or not).

\subsection*{\textit{Average Wages}}
Data, which describe the average earnings/wages in each municipality were taken into consideration. These are split into average wages per education, per position and per industry sector.

\subsection*{\textit{Municipal Finances \& Economy}}
Municipal finances such as the current revenue, current expenditure and capital expenditure were considered. In order to make the municipalities comparable the \textit{per capita} values were taken. Furthermore, the enterprise density and labor productivity are composed to the dataset.

\subsection*{\textit{Population Composition}}
The share of foreigners living in a municipality, the age composition, the young-dependency ratio, the ageing index, the population density and the potential sustainability index were used.


\subsection*{\textit{Political Affiliation}}
The result of the 2011 parliament election were taken into consideration.

}



\end{document}




















